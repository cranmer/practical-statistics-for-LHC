Unfortunately, Jeffreys's prior does not behave well in multidimensional problems.  Based on a similar information theoretic approach, Bernardo and Berger have developed the Reference priors~\cite{Berger:1992ys,Berger:1992vn,Berger:1989kx,Bernardo:1979uq} and the associated Reference analysis.  While attractive in many ways, the approach is fairly difficult to implement.  Recently,  there has been some progress within the particle physics context in deriving the reference prior for problems relevant to particle physics~\cite{Demortier:2010sn,Casadei:2011hx}.


\subsection{Likelihood Principle}


For those interested in the deeper and more philosophical aspects of statistical inference, the likelihood principle is incredibly interesting.  This section will be expanded in the future, but for now I simply suggest searching on the internet, the Wikipedia article, and Ref.~\cite{Birnbaum:1962}.  In short the principle says that all inference should be based on the likelihood function of the observed data.  Frequentist procedures violate the likelihood principle since p-values are tail probabilities associated to hypothetical outcomes (not the observed data).  Generally, Bayesian procedures and those based on the asymptotic properties of likelihood tests do obey the likelihood principle.  Somewhat ironically, the objective Bayesian procedures such as Reference priors and Jeffreys's prior can violate the likelihood principle since the prior is based on expectations over hypothetical outcomes.
