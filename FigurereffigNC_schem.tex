Figure~\ref{fig:NC_schematic} provides an overview of the classic Neyman construction corresponding to the left panel of Fig.~\ref{fig:neyman}.  The left panel of  Fig.~\ref{fig:neyman} is taken from the Feldman and Cousins's paper~\cite{Feldman:1997qc} where the parameter of the model is denoted $\mu$ instead of $\theta$.  For each value of the parameter $\mu$, the acceptance region in $x$ is illustrated as a horizontal bar.  Those regions are the ones that satisfy $T(\data)<k_\alpha$, and in the case of Feldman-Cousins the test statistic is the one of Eq.~\ref{eqn:tmu}.  This presentation of the confidence belt works well for a simple model in which the data consists of a single measurement $\data=\{x\}$.  Once one has the confidence belt, then one can immediately find the confidence interval for a particular measurement of $x$ simply by taking drawing a vertical line for the measured value of $x$ and finding the intersection with the confidence belt.


Unfortunately, this convenient visualization doesn't generalize to complicated models with many channels or even a single channel marked Poisson model where $\data=\{x_1,\dots,x_n\}$.  In those more complicated cases, the confidence belt can still be visualized where the observable $x$ is replaced with $T$, the test statistic itself.  Thus, the boundary of the belt is given by $k_\alpha$ vs. $\mu$ as in the right panel of Figure~\ref{fig:neyman}. The analog to the vertical line in the left panel is now a curve showing how the observed value of the test statistic depends on $\mu$.  The confidence interval still corresponds to the intersection of the observed test statistic curve and the confidence belt, which clearly satisfies $T(\data)<k_\alpha$.  For more complicated models with many parameters the confidence belt will have one axis for the test statistic and one axis for each model parameter.
