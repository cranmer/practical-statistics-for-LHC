\section{Conclusions}


It was a pleasure to lecture at the Cheile Gradistei school on High-Energy Physics and present the recent developments and practice in statistics for particle physics.  Quite a bit of progress has been made in the last few years in terms of statistical methodology, in particular the formalization of a fully frequentist approach to incorporating systematics, a deeper understanding of the look-elsewhere effect, the development of asymptotic approximations of the distributions important for particle physics, and in roads to Bayesian reference analysis. Furthermore, most of these developments are general purpose and can be applied across diverse models.   While those developments are interesting, the most important area for most physicists to devote their attention in terms of statistics is to improve the modeling of the data for his or her individual analysis.  With that perspective
