\section{Conceptual building blocks for modeling}


\subsection{Probability densities and the likelihood function}


This section specifies my notations and conventions, which I have chosen with some care.%



\int  f(x) \;dx\;= 1\;.


Figure~\ref{fig:hierarchy} establishes a hierarchy that is fairly general for the context of high-energy physics.  Imagine the search for the Higgs boson, in which the search is composed of several ``channels'' indexed by $c$.  Here a channel is defined by its associated event selection criteria, not an underlying physical process.  In addition to the number of selected events, $n_c$, each channel may make use of some other measured quantity, $x_c$, such as the invariant mass of the candidate Higgs boson. The quantities will be called ``observables'' and will be written in roman letters e.g. $x_c$. The notation is chosen to make manifest that the observable $x$ is  frequentist in nature.  Replication of the experiment many times will result in different values of $x$ and this ensemble gives rise to a \emph{probability density function} (pdf) of $x$, written $f(x)$, which has the important property that it is normalized to unity



f(x | \alpha) \;,


In the case of discrete quantities, such as the number of events satisfying some event selection, the integral is replaced by a sum.  Often one considers a parametric family of pdfs


read ``$f$ of $x$ given $\alpha$''  and, henceforth, referred to as a \emph{probability model} or just \emph{model}.  The parameters of the model typically represent parameters of a physical theory or an unknown property of the detector's response.  The parameters are not frequentist in nature, thus any probability statement associated with $\alpha$ is Bayesian.\footnote{Note, one can define a conditional distribution $f(x|y)$ when the joint distribution $f(x,y)$ is defined in a frequentist sense.}  In order to make their lack of frequentist interpretation manifest, model parameters will be written in greek letters, e.g.: $\mu, \theta, \alpha, \nu$.%
From the full set of parameters, one is typically only interested in a few: the \emph{parameters of interest}.  The remaining parameters are  referred to as \emph{nuisance parameters}, as we must account for them even though we are not interested in them directly.


While $f(x)$ describes the probability density for the observable $x$ for a single event, we also need to describe the probability density for a dataset with many events, $\data = \{x_1,\dots,x_{n}\}$.  If we consider the events as independently drawn from the same underlying distribution, then clearly the probability density is just a product of densities for each event.  However, if we have  a prediction that the total number of events expected, call it $\nu$, then we should also include the overall Poisson probability for observing $n$ events given $\nu$ expected.  Thus, we arrive at what statisticians call a marked Poisson model,
\begin{equation}
\label{Eq:markedPoisson}
\F(\data|\nu,\alpha) = \Pois(n|\nu) \prod_{e=1}^n f(x_e|\alpha) \; ,
\end{equation}
where I use a bold $\F$ to distinguish it from the individual event probability density $f(x)$.  In practice, the expectation is often parametrized as well and some parameters simultaneously modify the expected rate and shape, thus we can write $\nu\rightarrow\nu(\alpha)$.  In \texttt{RooFit} both $f$ and $\F$ are implemented with a \texttt{RooAbsPdf}; where \texttt{RooAbsPdf::getVal(x)} always provides the value of $f(x)$ and depending on \texttt{RooAbsPdf::extendMode()} the value of $\nu$ is accessed via \texttt{RooAbsPdf::expectedEvents()}.



\cancelto{\mathrm{Not ~True!}}{\int L(\alpha) \;d\alpha = 1}\; .


The \emph{likelihood function} $L(\alpha)$ is numerically equivalent to $f(x|\alpha)$ with $x$ fixed -- or $\F(\data|\alpha)$ with \data\ fixed.  The likelihood function should not be interpreted as a probability density for $\alpha$.  In particular, the likelihood function does not have the property that it normalizes to unity


It is common to work with the log-likelihood (or negative log-likelihood) function.  In the case of a marked Poisson, we have what is commonly referred to as an extended maximum likelihood~\cite{Barlow1990496}
\begin{eqnarray}
-\ln L( \alpha) &=& \underbrace{\nu(\alpha) - n \ln \nu(\alpha)}_{\rm extended~term} - \sum_{e=1}^n \ln f(x_e)   + \underbrace{~\ln n! ~}_{\mathrm{constant}}\; .
\end{eqnarray}
To reiterate the terminology, \emph{probability density function} refers to the value of $f$ as a function of $x$ given a fixed value of $\alpha$; \emph{likelihood function} refers to the value of $f$ as a function of $\alpha$ given a fixed value of $x$; and \emph{model} refers to the full structure of $f(x|\alpha)$.


Probability models can be constructed to simultaneously describe several channels, that is several disjoint regions of the data defined by the associated selection criteria.  I will use $e$ as the index over events and $c$ as the index over channels.  Thus, the number of events in the $c^{\rm th}$ channel is $n_c$ and the value of the $e^{\rm th}$ event in the $c^{\rm th}$ channel is $x_{ce}$.  In this context, the data is a collection of smaller datasets: \mbox{$\datasim=\{\data_1, \dots, \data_{c_{\rm max}}\}=\{\{x_{c=1,e=1}\dots x_{c=1,e=n_c}\}, \dots \{x_{c=c_{\rm max},e=1}\dots x_{c=c_{\rm max},e=n_{c_{\rm max}}} \}\}$}.  In \texttt{RooFit} the index $c$ is referred to as a \texttt{RooCategory} and it is used to inside the dataset to differentiate events associated to different channels or categories. The class \texttt{RooSimultaneous} associates the dataset $\data_c$ with the corresponding marked Poisson model.  The key point here is that there are now multiple Poisson terms.  Thus we can write the combined (or simultaneous) model 
\begin{equation}
\label{Eq:simultaneous}
\F_{\textrm{sim}}(\datasim|\alpha) = \prod_{c\in\rm channels} \left[ \Pois(n_c|\nu(\alpha)) \prod_{e=1}^{n_c} f(x_{ce}|\alpha) \right] \; ,
\end{equation}
remembering that the symbol product over channels has implications for the structure of the dataset.
