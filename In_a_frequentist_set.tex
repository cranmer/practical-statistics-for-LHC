In a frequentist setting, these allowed regions are called \textit{confidence intervals} or \textit{confidence regions}, and the parameter points outside them are considered excluded.  Associated with a confidence interval is a confidence level, i.e. the 95\% and 68\% confidence level in the two examples.  If we repeat the experiments and obtain different data, then these confidence intervals will change.  It is useful to think of the confidence intervals as being random  in the same way the data are random.  The defining property of a 95\% confidence interval is that it \textit{covers} the true value 95\% of the time.  


How can one possibly construct a confidence interval has the desired property, that it \textit{covers} the true value with a specified probability, given that we don't know the true value?  The procedure for building confidence intervals is called the Neyman Construction~\cite{Neyman}, and it is based on `inverting' a series of hypothesis tests (as described in Sec.~\ref{S:hypothesis test}).  In particular, for each value of $\vec\alpha$ in the parameter space one performs a hypothesis test based on some test statistic where the null hypothesis is $\vec\alpha$.  Note, that in this context, the null hypothesis is changing for each test and generally is not the background-only.  If one wants a 95\% confidence interval, then one constructs a series of hypothesis test with a size of 5\%.  The confidence interval $I(\data)$ is constructed by taking the set of parameter points where the null hypothesis is accepted. 
\begin{equation}
I(\data) = \left\{ \vec\alpha \middle |\, P(T(\data)>k_\alpha \,|\, \vec\alpha) < \alpha \right\} \;,
\end{equation}
where the final $\alpha$ and the subscript $k_\alpha$ refer to the size of the test.
Since a hypothesis test with a size of 5\% should accept the null hypothesis 95\% of the time if it is true, confidence intervals constructed in this way satisfy the defining property.  This same property is usually formulated in terms of \textit{coverage}.  Coverage is the probability that the interval will contain (cover) the parameter $\vec\alpha$ when it is true,
\begin{equation}
\textrm{coverage}(\vec\alpha) = P(\vec\alpha \in I\, |\, \vec\alpha) \; .
\end{equation}
The equation above can easily be mis-interpreted as the probability the parameter is in a fixed interval $I$; but one must remember that in evaluating the probability above the data $\data$, and, thus, the corresponding intervals produced by the procedure $I(\data)$, are the random quantities.  
Note, that coverage is a property that can be  quantified for any procedure that produces the confidence intervals $I$.  Intervals produced using the Neyman Construction procedure are said to ``cover by construction''; however, one can consider alternative procedures that may either under-cover or over-cover.  Undercoverage means that \mbox{$P(\vec\alpha \in I\, |\, \vec\alpha)$} is smaller than desired and over-coverage means that $P(\vec\alpha \in I\, |\, \vec\alpha)$ is larger than desired.  Note that in general coverage depends on the assumed true value $\vec\alpha$.


Since one typically is only interested in forming confidence intervals on the parameters of interest, then one could use the supremum $p$-value of Eq.~\ref{eqn:psup}.  This procedure ensures that the coverage is at least the desired level, though for some values of $\vec\alpha$ it may over-cover (perhaps significantly).  This procedure, which I call the `full construction',  is also computationally very intensive when $\vec\alpha$ has many parameters as it require performing many hypothesis tests.  In the naive approach where each $\alpha_p$ is scanned in a regular grid, the number of parameter points tested grows exponentially in the number of parameters.  There is an alternative approach, which I call the `profile construction'~\cite{Feldman,Cranmer:2005hi}
and which statisticians call an `hybrid resampling technique'~ \cite{Hybrid,Bodhi} that is approximate to the full construction, but typically has good coverage properties.  We return to the procedures and properties for the different types of Neyman Constructions later.
